\chapter{Gestures Identified as Appropriate for this Application –}